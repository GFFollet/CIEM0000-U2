
%-----General setup/geomtry-------------%
\documentclass[a4paper, 11pt,article,oneside]{memoir}%,openany
\setlrmarginsandblock{2.5cm}{2.5cm}{*}
\setulmarginsandblock{2.5cm}{2.5cm}{*}

\checkandfixthelayout
\usepackage[utf8]{inputenc} 
\setcounter{tocdepth}{2}
%--------------------------------------%

%------Misc Packages-------------------%
\usepackage[T1]{fontenc} 
\usepackage{calrsfs}
\usepackage{pifont}
\usepackage{amsmath}
\usepackage[dvipsnames]{xcolor}
\usepackage[breakable]{tcolorbox}
    \tcbuselibrary{skins}
    \tcbuselibrary{breakable}
\usepackage{stix}
\usepackage{graphicx}
\usepackage{parskip}
\usepackage[final]{hyperref}
\usepackage{listings}
\usepackage{float}
\usepackage{soul}
\usepackage{subcaption}
\usepackage{graphicx}
\usepackage{listings}

\bibliographystyle{plain}
\renewcommand\bibname{References} 

%-------------------------------------%

%----- Defining tcolorboxes-----------%

%Box for equations%
\newtcolorbox{eqbox}{
	colback=magenta!5!white,
	boxrule=0pt,
	frame hidden,
	sharp corners,
	enhanced,
	borderline east={2pt}{0pt}{magenta}}

%Box for equations with title%
\newtcolorbox{eqbox2}[3][]
{
  boxrule=0pt,
  fonttitle=\bfseries ,
  frame hidden,
  sharp corners,
  enhanced,
  borderline east ={2pt}{0pt}{magenta},
  colbacktitle  = white,
  colback  = magenta!5!white,
  coltitle = magenta,  
  title    = {#3},
  #1}
  
  
%Box for exercises%
\newtcolorbox{exbox}[3][]{
	breakable,
	colback=orange!5!white,
	boxrule=0pt,
	frame hidden,
	sharp corners,
	enhanced,
	borderline west={2pt}{0pt}{orange},
	coltitle = orange,
	title= {#3},
	#1,
	colbacktitle  = orange!5!white,
	fonttitle=\bfseries}
	
	
%Box for notes%
\newtcolorbox{note}{
	colback=cyan!5!white,
	boxrule=0pt,
	frame hidden,
	sharp corners,
	enhanced,
	borderline west={2pt}{0pt}{cyan},
	ragedleft}
%----------------------------------------------%

\begin{document}
\noindent
\setlength{\parindent}{0pt}

%---------TITLE PAGE--------------------------------%
\begin{titlingpage}
	\raggedleft % ]
	{\color{magenta}\rule{3 pt}{\textheight}} % Vertical line
	\hspace{0.05\textwidth} % Whitespace between the vertical line and title page text
	\parbox[b]{0.9\textwidth}{ % Paragraph box for holding the title page text, adjust the width to move the title page left or right on the page
	
		{\\\Huge\bfseries{ CIEM0000 U2.\\
		\textit{Mechanics for Civil Engineering}}}\\[3\baselineskip]  % Title
		{\large\textit{Complement to class notes }}\\[4\baselineskip] % Subtitle or further description
		 % Author name, lower case for consistent small caps\\
      	\\[4\baselineskip] 
		{\color{magenta}\textbf{Gabriel Follet\\g.f.follet@student.tudelft.nl}}
		
		\vspace{0.25\textheight} % Whitespace between the title block and the publisher
			{\noindent MSc CE TU Delft  v0.1- Q1 2023}} % Publisher and logo		

\end{titlingpage}
%----------------------------------------------%
\section*{Preface}
The following notes are based on the classes imparted at TU Delft during the first quarter of the year 2023.
CIEM0000 Interdisciplinary mechanics is the first course for all students enrolled in the Civil Engineering MSc.
 As the name implies, the content covered a wide range of mechanics problems. The course is organized in 14 topics, but I decide to forgo that structure and instead divide the course into 4 chapters.

The classes were imparted by the following professors
\begin{align*}
\textbf{Continuum mechanics - Basics: }&\text{Karel van Dalen \& Ad Reniers\&  Phil Vardon}\\
\textbf{Linear waves: }&\text{Robert Jan Labeur \& Karel van Dalen}\\
\textbf{Diffusion: }&\text{Phil Vardon}\\
\textbf{Amplitude dependant waves: }&\text{Robert Jan Labeur \& Ludovic Leclercq}
\end{align*} 

Feel free to either download, share and/or modified this document, the \textit{.tex} files are available  in\\ \textbf{{\color{magenta}\href{https://github.com/GFFollet/CIEM0000-U2.git}{https://github.com/GFFollet/CIEM0000-U2.git}}}\\
If this document helped you, consider fixing any mistake you may have found and making a commit to the repo, from simple spelling errors (I'm sure there will be a lot of mistakes specially spelling) 

\newpage
\tableofcontents
\newpage



\chapter{Continuum Mechanics: Basics}
Every continuum mechanics problem can be reduced to three basic steps

\begin{enumerate}
\item Kinematics
\item Action-Deformation
\item Equilibrium
\end{enumerate}
	

\section{Kinematics}
Let's consider the following expression
\begin{align*}
\varphi(\boldsymbol{X})=\boldsymbol{X}+\boldsymbol{u(X)}
\end{align*}
Where 
\begin{align*}
\boldsymbol{X}:& \text{Initial position}\\
\boldsymbol{u (X)}:& \text{Displacement field}\\
\varphi(\boldsymbol{X}):& \text{Final position}
\end{align*}
Then we can define 
\begin{eqbox2}{red}{Lagrangian Strain Tensor}
\begin{align*}
\boldsymbol{E}:=\dfrac{1}{2}\left(\nabla\boldsymbol{u}+ \nabla\boldsymbol{u}^T+\nabla\boldsymbol{u}\nabla\boldsymbol{u}^T \right)
\end{align*}
\end{eqbox2}
For \textit{small} displacement gradients
\begin{eqbox2}{red}{Cauchy Strain Tensor}
\begin{align*}\label{tensor}
\boldsymbol{E} \simeq \boldysmbol{\varepsilon}:=&\dfrac{1}{2}\left(\nabla\boldsymbol{u}+\nabla\boldsymbol{u}^T \right)
\end{align*}
\end{eqbox2}
\begin{note}
In civil engineering, the Cauchy Tensor (Engineering tensor) is almost always used. See \cite{ICE2114} for rigorous definition/proof.
\end{note}
In matrix form
\begin{align*}
\boldsymbol{\varepsilon}=&\begin{bmatrix}
\frac{\partial u_x}{\partial x}&\frac{1}{2}\left(\frac{\partial u_x}{\partial x}+\frac{\partial u_y}{\partial x}\right)&\frac{1}{2}\left(\frac{\partial u_x}{\partial z}+\frac{\partial u_z}{\partial x}\right)\\
\frac{1}{2}\left(\frac{\partial u_y}{\partial x}+\frac{\partial u_x}{\partial y}\right)&\frac{\partial u_y}{\partial y}&\frac{1}{2}\left(\frac{\partial u_y}{\partial z}+\frac{\partial u_z}{\partial y}\right)\\
\frac{1}{2}\left(\frac{\partial u_z}{\partial x}+\frac{\partial u_x}{\partial z}\right)&\frac{1}{2}\left(\frac{\partial u_z}{\partial y}+\frac{\partial u_y}{\partial z}\right)&\frac{\partial u_z}{\partial z}
\end{bmatrix}
\end{align*}
We can also define the shear strain angle as
\begin{align*}
\gamma_{i,j}=\alpha+\beta\\
\end{align*} 
Where: 
\begin{align*}
\tan (\alpha)=\dfrac{\frac{\partial u_j}{\partial i}}{1+\frac{\partial i}{\partial u_i}}\qquad \qquad 
\tan (\beta)=\dfrac{\frac{\partial u_i}{\partial j}}{1+\frac{\partial j}{\partial u_j}}
\end{align*}
For \textit{small angles}
and \textit{small displacement gradients}, the shear strain angle is
\begin{align*}
\gamma_{i,j}=\gamma_{j,i}=\frac{\partial u_i}{\partial j}+\frac{\partial u_j}{\partial i}
\end{align*}
Then we can express the Cauchy Strain Tensor as
\begin{eqbox}
\begin{align*}
\boldsymbol{\varepsilon}=&\begin{bmatrix}
\varepsilon_{xx}&\dfrac{\gamma_{xy}}{2}&\dfrac{\gamma_{xz}}{2}\\
\dfrac{\gamma_{yx}}{2}&\varepsilon_{yy}&\dfrac{\gamma_{yz}}{2}\\
\dfrac{\gamma_{zx}}{2}&\dfrac{\gamma_{zy}}{2}&\varepsilon_{zz}
\end{bmatrix}
\end{align*}
\end{eqbox}
Because the Cauchy Strain Tensor is symmetrical we can express it as a vector using Voigt notation
\begin{align*}
\varepsilon=\begin{bmatrix}
\varepsilon_{xx}\\
\varepsilon_{yy}\\
\varepsilon_{zz}\\
\gamma_{xy}\\
\gamma_{xz}\\
\gamma_{yz}\\
\end{bmatrix}
\end{align*}
In some cases it may be useful to compute the rigid body rotation 
\begin{align*}\label{Rigid_body_rotation}
\boldsymbol{\theta}=\nabla\times\varepsilon=&\begin{bmatrix}
0&\frac{1}{2}\left(\frac{\partial u_x}{\partial x}-\frac{\partial u_y}{\partial x}\right)&\frac{1}{2}\left(\frac{\partial u_x}{\partial z}−\frac{\partial u_z}{\partial x}\right)\\
-\frac{1}{2}\left(\frac{\partial u_y}{\partial x}-\frac{\partial u_x}{\partial y}\right)&0&\frac{1}{2}\left(\frac{\partial u_y}{\partial z}-\frac{\partial u_z}{\partial y}\right)\\
-\frac{1}{2}\left(\frac{\partial u_z}{\partial x}-\frac{\partial u_x}{\partial z}\right)&-\frac{1}{2}\left(\frac{\partial u_z}{\partial y}-\frac{\partial u_y}{\partial z}\right)&0
\end{bmatrix}
\end{align*}
Analogous to the Cauchy Strain Tensor, we define the Strain Rate
Tensor


\begin{eqbox2}{red}{Cauchy Strain Rate Tensor}
\begin{align*}
\boldysmbol{\dot{\varepsilon}}:=&\dfrac{1}{2}\left(\nabla\boldsymbol{\dot{u}}+\nabla\boldsymbol{\dot{u}}^T \right)
\end{align*}
\end{eqbox2}
Then, the rotation of a fluid element is 
\begin{align*}
\boldsymbol{\Omega}=&\nabla \times \boldsymbol{v}
\end{align*}
\subsection{Basic tensor properties}
From linear algebra we know that we can change basis such that
\begin{align*}
\boldsymbol{\varepsilon}'=Q\boldsymbol{\varepsilon}Q^T
\end{align*} 
If we use a unit vector ($\hat{n}$) instead of the basis matrix, ($Q$) we can obtain the deformation in the direction $\hat{n}$ we obtain the projection in that direction.

\begin{equation*}
\varepsilon_{axial}(\hat{n})=\hat{n}^T\varepsilon\hat{n}
\end{equation*}
If we want to obtain the principal directions, we need to solve the following minimization problem 
\begin{align*}
\text{max/min}\quad& (\hat{N}^T\epsilon\hat{N})\\
\text{such that}\quad& \vert\hat{N}\vert=1
\end{align*}
Using the Lagrange multipliers method, we get the problem is equivalent to an eigenvalue problem
\begin{eqbox}
\begin{equation*}
\left(\boldsymbol{\varepsilon-\lambda I} \right)N=0 \qquad \text{where} \vert N\vert=1
\end{equation*}
\end{eqbox}
As the name suggests, the invariants of a tensor are properties of the tensor that do not depend on the basis, the most common ones are
\begin{align*}
I_1 =&\,\text{tr}(\boldsymbol{\varepsilon})\\
I_2=\, &\frac{1}{2}\left(\text{tr}(\boldsymbol{\varepsilon})^2-\text{tr}(\boldsymbol{\varepsilon^2})\right)\\
I_3=\, &\text{det}(\boldsymbol{\varepsilon})
\end{align*}
\begin{note}
The fact that we can diagonalize the strain/stress tensor, means that we can find a frame of reference (or direction) in which there is no shear or only shear. In many cases we can just reduce the problem to principal stresses.\\
Tensor invariants are very useful in soil mechanics, but there we consider the Effective Stress Tensor and its invariants ($J_1,J_2,J_3$), see \cite{ICE2614} for details)
\end{note}






\section{Action-Deformation}
In linear materials   
\begin{equation*}\label{HookLaw}
\boldsymbol{\sigma}=\boldsymbol{A}\boldsysbol{\varepsilon}
\end{equation*}
For an isotropic linear-elastic material, the expression can be simplified to

\begin{eqbox2}{red}{Generalized Hook's Law}\
\begin{equation*}
\boldsymbol{\sigma}=\boldsymbol{C}\boldsysbol{\varepsilon}=\frac{E}{(1+\nu)(1-2\nu)}
\begin{bmatrix}
1-\nu&\nu&\nu&0&0&0\\
\nu&1-\nu&\nu&0&0&0\\
\nu&\nu&1-\nu&0&0&0\\
0&0&0&1-2\nu&0&0\\
0&0&0&0&1-2\nu&0\\
0&0&0&0&0&1-2\nu\\
\end{bmatrix}\boldsysbol{\varepsilon}
\end{equation*}
\end{eqbox2}
Where $\boldsymbol{C}$ is the Stiffness matrix.
\begin{note}
In most civil engineering  problems Hook's Law is sufficiently accurate, except  maybe in 
\begin{itemize}
\item (Foliated) Rock masses, the isotropic hypothesis is not accurate, we may need up to 21 elastic constant. Using ultrasonic/seismic methods we can determine the elastic constant in multiple direction, see \cite{CIVIL448} for detail in how to obtain this parameters.
\item Soil Mechanics, in particular when the clay content is important it is necessary to use elasto-plastic models, eg. Cam Clay in \cite{ICE2614}.
\item Earthquake Engineering, even without going for the full nonlinear modeling, implementing a hybrid approach using plastic hinges (elastoplasticity) may reduce the complexity but yield accurate enough results to design,(obviously we need the nonlinear modeling to validate the design but using an hybrid approach can reduce time and money (same for the design of metallic dissipation devices), see\cite{CIVIL467} and \cite{ICE3753} for introduction/basics.
\end{itemize}
\end{note}
It may be useful to include the effect of temperature and initial stress (prestressed structures)
\begin{eqbox}
\begin{equation*}
\boldsymbol{\sigma}=\boldsymbol{C}(\boldsysbol{\varepsilon-\varepsilon_0})+\sigma_0
\end{equation*}
\end{eqbox}
Likewise, we can also express this relationship in the other way using the Flexibility matrix ($\boldsymbol{S}$)
\begin{equation*}
\boldsymbol{\varepsilon}=\boldsymbol{S}\boldsysbol{\sigma}
\end{equation*}

\begin{eqbox}
\begin{equation*}
\boldsymbol{S}=\frac{1}{E}
\begin{bmatrix}
1&-\nu&-\nu&0&0&0\\
-\nu&1&-\nu&0&0&0\\
-\nu&-\nu&1&0&0&0\\
0&0&0&(1+\nu)&0&0\\
0&0&0&0&(1+\nu)&0\\
0&0&0&0&0&(1+\nu)\\
\end{bmatrix}
\end{equation*}
\end{eqbox}
The elastic properties of the material can be also expressed with the Lamé constants.
\begin{gather*}
\lambda=\frac{E\nu}{(1-\nu)(1-2\nu)}\\
\mu=G=\frac{E}{2(1+\nu)}
\end{gather*}

\section{Balance Laws}
Let's consider an intensive property $f$, then
\begin{eqbox}
\begin{equation*}
F=\int_\mathcal{B} f\rho \,dV
\end{equation*}
\end{eqbox}
Where $F$ is the equivalent extensive property, then we can express multiple conservation laws through the following equation 
\begin{eqbox2}{red}{Master Balance Principle}
\begin{equation*}
\frac{DF}{Dt}=\frac{D}{Dt}\int_\mathcal{B} f(x,t)\, dV=\int_{\partial \mathcal{B}}\phi(x,t,\hat{n})\,dS +\int_\mathcal{B} \Sigma(x,t) \,dV
\end{equation*}
\end{eqbox2}
The Surface($\phi$) and Volume($\Sigma$) density terms represent the surface and volumetric flows. 
\begin{note}
The previous equation is extremely general and not really used in practice , but I think is necessary to at least know the proper formulation, see \cite{ICE2114} for full proof.
\end{note}
Then, we can construct multiple conservation laws, choosing different values of $f,\phi,\sigma$.
\begin{eqbox2}{red}{Conservation of Mass}
\begin{align*}
M=\int_V \rho \,dV
\end{align*}
\begin{equation*}
\frac{DM}{Dt}=\frac{D}{Dt}\int_\Omega \rho(x,t) \,dV=0
\end{equation*}
\end{eqbox2}
\begin{eqbox2}{red}{Conservation of Linear Momentum}
\begin{align*}
\boldsymbol{p}=\int_V \rho v \,dV
\end{align*}
\begin{equation*}
\frac{D\boldsymbol{p}}{Dt}=\frac{D}{Dt}\int_\Omega \rho v\, dV=\int_{\partial \Omega}\boldsymbol{t}\,dS +\int_\Omega \boldsymbol{b} \,dV
\end{equation*}
\end{eqbox2}
\begin{eqbox2}{red}{Conservation of Angular Momentum}
\begin{align*}
\boldsymbol{L}=\int_V  \boldysmbol{r}\times \rho v \,dV
\end{align*}
\begin{equation*}
\frac{D\boldsymbol{L}}{Dt}=\frac{D}{Dt}\int_\Omega \boldysmbol{r}\times\rho v\, dV=\int_{\partial \Omega}\boldysmbol{r}\times\boldsymbol{t}\,dS +\int_\Omega \boldysmbol{r}\times\boldsymbol{b} \,dV
\end{equation*}
\end{eqbox2}
\begin{eqbox2}{red}{First Law of Thermodynamics}
\begin{align*}
\boldsymbol{E}=\int_V \rho e \,dV
\end{align*}
\begin{equation*}
\frac{DE}{Dt}=\frac{D}{Dt}\int_\Omega \left(gz+\frac{v^2}{2}+\hat{u}(T)+\frac{P}{\rho}\right)\rho\, dV=\int_{\partial \Omega}\phi(x,t,\hat{n})\,dS +\int_\Omega \Sigma(x,t) \,dV
\end{equation*}
\vspace{0.2cm}
In gases, enthalpy is used instead 
\begin{align*}
\hat{h}(T)=\hat{u}(T)+\frac{P}{\rho}
\end{align*}
\end{eqbox2}

\section{Simplifications/Reduction of balance laws}
\subsection{Simplified differential form}
As we previously stated, the master conservation principle is not used in practice instead we use simplified  equation in differential form.
\begin{eqbox2}{}{Simplified  Master Balance Principle in  Differential form}
\begin{gather*}
\frac{\partial Q}{\partial t}=-\frac{\partial q_x}{\partial x}-\frac{\partial q_y}{\partial y}-\frac{\partial q_z}{\partial z}
\end{gather*}
Which essentially translates to , \textit{The variation of $Q$ is equal to the fluxes through the boundaries}
\end{eqbox2}
Using this simplified formulation, some commonly used conservation laws are
\begin{eqbox2}{red}{Simplified Mass Conservation}
If we choose 
\begin{gather*}
Q=\rho \\
q_i=\rho v_i
\end{gather*}
\begin{gather*}
\frac{\partial \rho}{\partial t}=-\frac{\partial \rho v_x}{\partial x}-\frac{\partial \rho v_y}{\partial y}-\frac{\partial \rho v_z}{\partial z}
\end{gather*}
\end{eqbox2}
\begin{eqbox2}{red}{Simplified Conservation of Momentum}
If we choose 
\begin{gather*}
Q=\rho v_i \\
q_i=\rho v_i v_j
\end{gather*}

\begin{align*}
\frac{\partial \rho v_x}{\partial t}=-\frac{\partial v_x( \rho v_x)}{\partial x}-\frac{\partial v_y( \rho v_x)}{\partial y}-\frac{\partial v_z( \rho v_x)}{\partial z}+\left(\frac{\sigma_{xx}}{\partial x}+\frac{\sigma_{yx}}{\partial y}+\frac{\sigma_{zx}}{\partial z}\right)+f_x\\
\frac{\partial \rho v_y}{\partial t}=-\frac{\partial v_x( \rho v_y)}{\partial x}-\frac{\partial v_y( \rho v_y)}{\partial y}-\frac{\partial v_z( \rho v_y)}{\partial z}+\left(\frac{\sigma_{xy}}{\partial x}+\frac{\sigma_{yy}}{\partial y}+\frac{\sigma_{zy}}{\partial z}\right)+f_y\\
\frac{\partial \rho v_z}{\partial t}=-\frac{\partial v_x( \rho v_z)}{\partial x}-\frac{\partial v_y( \rho v_z)}{\partial y}-\frac{\partial v_z( \rho v_z)}{\partial z}+\left(\frac{\sigma_{xz}}{\partial x}+\frac{\sigma_{yz}}{\partial y}+\frac{\sigma_{zz}}{\partial z}\right)+f_z\\
\end{align*}

\end{eqbox2}

\begin{eqbox2}{}{Linearize Momentum Equation}
If we consider that the density is constant,and that the velocity is 
\begin{gather*}
v_i=\frac{\partial u_i}{\partial t}
\end{gather*}
If we decide that the spacial-temporal cross term are negligible
\begin{gather*}
\frac{\partial v_j( \rho v_j)}{\partial x_i}=\rho\frac{\partial}{\partial x_i}\left(\frac{\partial  u_i \partial  u_j}{\partial t^2}\right)\approx 0
\end{gather*}
We get the linearize momentum equation
\begin{align*}
\rho \frac{\partial^2u_x}{\partial t^2}=
&\frac{\partial\sigma_{xx}}{\partial x}+\frac{\partial\sigma_{yx}}{\partial y}+\frac{\partial\sigma_{zx}}{\partial z}+f_x\\
\rho \frac{\partial^2u_y}{\partial t^2}=&\frac{\partial\sigma_{xy}}{\partial x}+\frac{\partial\sigma_{yy}}{\partial y}+\frac{\partial\sigma_{zy}}{\partial z}+f_y\\
\rho \frac{\partial^2u_z}{\partial t^2}=&\frac{\partial\sigma_{xz}}{\partial x}+\frac{\partial\sigma_{yz}}{\partial y}+\frac{\partial\sigma_{zz}}{\partial z}+f_z
\end{align*}
\end{eqbox2}


\begin{eqbox2}{red}{Laplace Fluids Law}\label{LaplaceFlow}
\begin{align*}
k\nabla^2h=0
\end{align*}
\end{eqbox2}
\subsection{Dimensional reduction}
In many cases, we don't really care about the response of the system in all direction. The magnitude of the response in some direction could be negligible or because of symmetries in the system we could expect them to be, nevertheless, in many cases we are interested in 2D system or even in the 1D. 

\begin{eqbox2}{}{Plane Stress}
Assuming that there are no stress in the out-of-plane direction,
\begin{gather*}
\begin{bmatrix}
\varepsilon_{xx}\\
\varepsilon_{zz}\\
\varepsilon_{zx}\\
\end{bmatrix}=\frac{1}{E}
\begin{bmatrix}
1&-\nu&0\\
-\nu&1&0\\
0&0&1+\nu\\
\end{bmatrix}
\begin{bmatrix}
\sigma_{xx}\\
\sigma_{zz}\\
\sigma_{zx}\\
\end{bmatrix}\\
\varepsilon_{yy}=\frac{-\nu}{E}(\sigma_{xx}+\sigma_{zz})
\end{gather*}
Alternatively
\begin{gather*}
\begin{bmatrix}
\sigma_{xx}\\
\sigma_{zz}\\
\sigma_{zx}\\
\end{bmatrix}=\frac{E}{(1+\nu)(1-\nu)}
\begin{bmatrix}
1&\nu&0\\
\nu&1&0\\
0&0&1-\nu\\
\end{bmatrix}
\begin{bmatrix}
\varepsilon_{xx}\\
\varepsilon_{zz}\\
\varepsilon_{zx}\\
\end{bmatrix}\\
\end{gather*}
At first glance, everything seems consistent, but if we consider the linearized momentum equation 
\begin{gather*}
\rho \frac{\partial^2u_y}{\partial t^2}=\frac{\partial\sigma_{xy}}{\partial x}+\frac{\partial\sigma_{yy}}{\partial y}+\frac{\partial\sigma_{zy}}{\partial z}+f_y\\ \\
\rho \frac{\partial^2u_y}{\partial t^2}=0
\end{gather*}
However that implies that there is no acceleration, which is only true in the static case
\end{eqbox2}
\begin{eqbox2}{}{Plane Strain}
Assuming that there are no strain in the out-of-plane direction,
\begin{gather*}
\begin{bmatrix}
\varepsilon_{xx}\\
\varepsilon_{zz}\\
\varepsilon_{zx}\\
\end{bmatrix}=\frac{1+\nu}{E}
\begin{bmatrix}
1-\nu&-\nu&0\\
-\nu&1-\nu&0\\
0&0&1\\
\end{bmatrix}
\begin{bmatrix}
\sigma_{xx}\\
\sigma_{zz}\\
\sigma_{zx}\\
\end{bmatrix}\\
\end{gather*}
Alternatively
\begin{gather*}
\begin{bmatrix}
\sigma_{xx}\\
\sigma_{zz}\\
\sigma_{zx}\\
\end{bmatrix}=\frac{E}{(1+\nu)(1-2\nu)}
\begin{bmatrix}
1-\nu&\nu&0\\
\nu&1-\nu&0\\
0&0&1-2\nu\\
\end{bmatrix}
\begin{bmatrix}
\varepsilon_{xx}\\
\varepsilon_{zz}\\
\varepsilon_{zx}\\
\end{bmatrix}\\ \\
\sigma_{yy}=\frac{\nu E}{(1+\nu)(2-2\nu)}(\varepsilon_{xx}+\varepsilon_{zz})
\end{gather*}
Which is consistent and does not presuppose the static case as plain stress does.
\end{eqbox2}

Using plane strain/stress and some assumptions we can model very common cases such as :\\

{\color{magenta}{\textbf{Axially loaded bar}}}\\
Considering bar biaxially loaded, if we integrate over the cross section we get

\begin{gather*}
\rho A\frac{\partial^2\bar{u}_x}{\partial t^2}=EA\frac{\partial^2 \bar{u}_x}{\partial x^2}+q_x
\end{gather*}


{\color{magenta}{\textbf{Shear layer}}}\\
Lets consider a layer loaded in pure shear (double plane strain), such that there is only shear deformation.
Then replacing in the linearized equation of motion we get 
\begin{gather*}
\rho \frac{\partial^2 u_x}{\partial t^2}=G\frac{\partial^2 u_x}{\partial z}+f_x
\end{gather*}


{\color{magenta}{\textbf{Diaphragm wall}}}\\
Lets consider a wall loaded vertically, such that it can be considered to be in plane stress and plane strain, integrating over the thickness of wall ($d$)
\begin{gather*}
\rho d \frac{\partial^2 \bar{u}_x}{\partial t^2}=\frac{Ed}{1-\nu^2}\frac{\partial^2 \bar{u}_x}{\partial z}+p_z
\end{gather*}

\section{Case Study: Darcy's law and potential flow}
Let consider the conservation of mass (volume, assuming that the density is constant) for a control volume we get  
\begin{eqbox2}{red}{Continuity Equation}
\begin{align*}
\frac{\partial \rho}{\partial t}=&-\nabla q\\
\frac{\partial \rho}{\partial t}=&-\rho \nabla v\\
\end{align*}
Considering that there is no storage capacity  and steady state we get
\begin{gather*}
\rho \nabla v=0
\end{gather*}
\end{eqbox2}

Lets remember that the simplest model for flow is the potential flow, also known as Darcy's Law, it says that "\textit{water flows between energy gradients}"

Lets define the potential (energy) or head as
\begin{gather*}
\Psi=h=z+\frac{p}{\rho g}+\frac{1}{2}\frac{U^2}{2g}\\
\end{gather*}
In water flow through soil, the velocity term is negligible
$$ h\approx z+\frac{p}{\rho g}$$
Then mathematically we can express potential flow as 
\begin{eqbox2}{}{Darcy's Law}
\begin{gather*}
q=-k\nabla h
\end{gather*}
$k$: Hydraulic conductivity 
\end{eqbox2} 
Combining Darcy's Law with the Continuity Equation we get 
\begin{eqbox2}{}{Laplace's Fluids Law}
\begin{gather*}
k\nabla^2h=0
\end{gather*}
\end{eqbox2}
We know that the water flow between the \textit{soil grains} then we can estimate the flow velocity as 
\begin{gather*}
v=\frac{q}{n}
\end{gather*}
With $n: $ Porosity
\begin{note}
In rocks, we have to be careful about porosity vs connected porosity and permeability vs hydraulic conductivity.\end{note} 
\section{Case Study: Simplified flow through deformable soil}
Lets consider the case of fully saturated soil, and no chemical of mechanical interaction/degradation of the soil due to the flow/presence of water. We also know that soil fails in term of effective stresses, then it is convenient to define the effective stress tensor as
\begin{eqbox2}{}{Effective stress} 
\begin{align*}
\sigma'=\sigma-\boldsymbol{I}u_w
\end{align*}
\end{eqbox2}
As we mentioned before, in soil mechanics it is useful to consider the invariants of the effective stress tensor, which are analogous to the stress tensor's but called $J_1$ $J_2$ and $J_3$ respectively.\\
For the continuity equation we previously assumed that $ \Sigma=0$) but from the first chapter we know that the medium can deform, therefore the control volume can deform and we have to include the change of volume in the continuity equation
\begin{eqbox2}{red}{Deformable medium - Continuity Equation}
For permanent flow
\begin{align*}
\frac{\varepsilon_v}{\partial t}=-\frac{k}{\rho g} \nabla^2 h
\end{align*}
\end{eqbox2}
With 
$I_1=\varepsilon_v$

Lets consider the case of 1D consolidation due to an external stress,considering that the soil properties are constant

\begin{gather*}
\frac{\sigma_{zz}'}{\partial x}=-\frac{\partial p_e}{\partial z1}\\
\frac{E(1-\nu)}{(1-2\nu)(1+\nu)}\frac{\partial \varepsilon_{zz}}{\partial z}=-\frac{\partial p_e}{\partial z}\\
 \frac{\partial }{\partial t}\left(\int\frac{E(1-\nu)}{(1-2\nu)(1+\nu)}\frac{\partial \varepsilon_{zz}}{\partial z}\, dz\right) =\frac{\partial}{\partial t} \left(\int -\frac{\partial p_e}{\partial z}\, dz\right) \\
\frac{E(1-\nu)}{(1-2\nu)(1+\nu)}\frac{\partial \varepsilon_{zz}}{\partial t}=-\frac{\partial p_e}{\partial t}\
\end{gather*}
In the 1D case the continuity equation in deformable medium simplifies to , $\varepsilon_v=\text{tr}(\varepsilon)=\varepsilon_{zz}$, replacing in the previous equation we get 
\begin{eqbox2}{}{Therzaghi's time rate consolidation}
For saturated clays soils, the rate of secondary consolidation can be estimated from
\begin{gather*}
\frac{\partial p_e}{\partial t}=C_v \frac{\partial^2 p_e}{\partial z^2}
\end{gather*}
 $C_v=\dfrac{k}{m_v\rho g}$\\
 $m_v=\dfrac{(1-2\nu)(1+v)}{E(1-\nu)}$\\
 
 The equation can be solved analytically with the tools that we will learn in the {\color{magenta}Diffusion Section}, but in practice it is usually solved using graphical methods. See \cite{ICE2604} for detail of graphical methods, load ramp correction, pre(over)consolidated clays consideration, and estimation of $C_v$ from field data.
\end{eqbox2}
\section{Case Study: Flow to a well}
Let's consider
\begin{itemize}
\item Saturated soils
\item Single fluid, i.e. water
\item Isotropic soil
\item Water is incompressible
\end{itemize}




\subsection{Flow through 1D soil layer}


The soil has hydraulic conductivity $k$,and the potential in the boundaries are $P_1$ and $P_2$, therefore it is a Dirichlet problem.

Using the Laplace Flow equation and considering that the problem is 1D
\begin{align*}
k\frac{\partial^2 h}{\partial x^2}=0\qquad \text{and}\qquad h|^{0=P_1}_{L=P_2}
\end{align*}
Integrating
\begin{gather*}
\int k\frac{\partial^2 h}{\partial x^2}=\,0\\
k\frac{\partial h}{\partial x^2}+C_1=\,0\\
\int k\frac{\partial h}{\partial x^2}+C_1=\,0\\ 
kh+C_1x+C_2=\,0
\end{gather*}
Applying Boundary conditions
\begin{align*}
h(x)=\left(\frac{P_2-P_1}{L}\right)x+P_1
\end{align*}

\subsection{Flow through 1D soil, with discontinuity in soil properties}\\
Because the hydraulic conductivity is discontinuous in the layer, it is easier to consider the problem as 2 problems of 1D isotropic soil, but applying boundary conditions in the interface as to preserve the conservation of mass principle.
\begin{note}
If we did not care about the distribution, but only about the global behavior, we could have used the equivalent hydraulic conductivity.
\begin{gather*}
k_{eq}^h=\frac{\Sigma k_id_i}{\Sigma d_i}\\
k_{eq}^v=\frac{\Sigma d_i}{\Sigma \frac{d_i}{k_i}}\
\end{gather*}
\end{note}
Then, similarly to the last problem 
\begin{align*}
k_1\frac{\partial^2 h_1(x)}{\partial x^2}=&\,0 \\
k_2\frac{\partial^2 h_2(x)}{\partial x^2}=&\,0 \\
h_1(0)=&\,P_1\\
h_2(L)=&\,P_2 \\
k_1\frac{\partial h_1(L/2)}{\partial x}=&\,k_2\frac{\partial h_2(L/2)}{\partial x}
\end{align*}
\begin{note}
Depending on the complexity of the situation, it may be useful to define local coordinates systems to simplify the algebra, for example
\begin{align*}
x_1'=&\,x\\
x_2'=&\,x+\frac{L}{2}
\end{align*}
However, in the previous exercise because $k$ was uniform in each layer, it is just easier to use the result from question 1, where we know the pressure distribution in terms of  the boundary conditions, the problem can be easily reduced to 
\begin{align*}
h_1'(x_1)=\frac{P_3-P_1}{L}x+P_1\\
h_2'(x_2)=\frac{P_2-P_3}{L}x+P_3\\
k_1\frac{\partial h_1'(0)}{\partial x_1}=k_2\frac{\partial h_2'(0)}{\partial x_2}
\end{align*}
\end{note}


\subsection{Vertical 1D flow through isotropic layer}\\
Considering the boundary condition of no flux in the bottom and no pressure in the top, and that the energy is 
\begin{align*}
h(z)=gz+\frac{v(z)^2}{2}+\frac{P(z)}{\rho(z)}\approx gz+\frac{P(z)}{\rho}\
\end{align*}
\begin{note}
It is necessary to pay attention to the unit of energy, dependent on the situation, it may be better to express the energy in terms of height , pressure or just energy.\\
In addition, the \textit{no-pressure} boundary condition  refers to \textit{relative pressure}, because
\begin{align*}
P_{top}=&P_{atm}+0\\
P_{bottom}=&P_{atm}+P_{bottom}
\end{align*}
But atmospheric pressure usually affects both boundary condition. 	

We know from problem 1 that the solution to 1D flow  is 
\begin{align*}
kh+C_1x+C_2=& \,0
\end{align*}
Then applying Boundary conditions
\begin{align*}
h(z_{top})=0\\
\frac{\partial h(z_{bottom})}{\partial z}=0
\end{align*}
We get 
\begin{align*}
h(z)=\rho g (z_{top}-z)
\end{align*}
\begin{note}
In problems involving atmospheric transport, we need to remember that $g(z)$ and $\rho (z)$. See \cite{ICH2304} for examples.
\end{note}
\end{exbox}
\subsection{Radial flow}\\
The Laplace Flow equation  works in any dimension and coordinates system\footnote{ See \cite{MAT1620} for the Laplace operator in any coordinate systems}.
\begin{align*}
k\left(\frac{\partial^2h}{\partial r^2}+\frac{1}{r}\frac{\partial h}{\partial r}+\frac{1}{r^2}\frac{\partial h^2}{\partial \theta^2}\right)=0
\end{align*}
Because the situation is radially symmetric($\frac{\partial h^2}{\partial \theta^2}=0$)
\begin{gather*}
k\left(\frac{\partial^2h}{\partial r^2}+\frac{1}{r}\frac{\partial h}{\partial r}\right)=0\\
h=C_1\ln r+C_2
\end{gather}


\section{Case study: Static deformation of a bar with distributed
elastic foundation}
Let's consider an elastic pile foundation, with
\begin{itemize}
\item Axial stiffness: $EA$
\item Distributed spring stiffness: $\chi$, accounting for soil resistance 
\item Discrete spring: $K$, accounting for soil resistance at the tip
\item External force: $F_0$

\end{itemize}

\textbf{Kinematics}\\
It is trivial to see that the displacement field of the bar is equal to the displacement field of the later soil 
\begin{align*}
u_z^{bar}=u_z^{soil}=u_z
\end{align*}
And also that the displacement in the tip is equal to the deformation of the soil under the tip
\begin{align*}
\frac{du_z(L)}{dz}=\delta_{tip}
\end{align*}

\textbf{Action-Deformation}\\
We have three different action-deformation relationships
\begin{itemize}
\item Bar\\
$$\sigma_{bar}=EA\left(\frac{du_z}{dz}\right)$$

\item \textit{Skin force} (Soil) 
$$\sigma_{soil}=\chi u_z$$
\begin{note}
Strictly, $ \sigma_{soil}=\chi\frac{du_z'}{dz'}$, but the deformation for every \textit{spring} is $\frac{du_z'}{dz'}=u_z-u_z^0$, and in the continuum $u_z^0=0$, then the deformation of every \textit{spring} is just the displacement of the bar at that point}\\
\end{note}
\item  \textit{Tip force} (Soil)\\
$$F_{tip}=K\left(\frac{du_z(L)}{dz}\right)$$
\end{itemize}
\textbf{Equilibrium}\\
From the conservation of momentum in a 1D static situation
\begin{align*}
\frac{d}{dz}\left( EA\frac{du_z}{dz}\right) +q_z=0\\
\frac{d}{dz}\left( EA\frac{du_z}{dz}\right) -\chi u_z=0
\end{align*}
We have the stiffness of the soil and the pile expressed in different forms, it may be useful to solve the problem in terms of the ratio of soil/bar stiffness($\kappa^2=\frac{\chi}{EA}$)
\begin{align*}
\frac{d^2u_z}{dz^2} -\kappa^2 u_z=0
\end{align*}
\textbf{General solution }\\
The solution is 
\begin{align*}
u_z(z)=C_1 e^{\kappa z}+C_1 e^{-\kappa z}
\end{align*}
\textbf{Boundary Conditions}\\
The boundary conditions are the deformation in the top and bottom
\begin{align*}
\left( EA\frac{du_z}{dz}\right)\left|_{z=0}&=-F_0\\
\left( EA\frac{du_z}{dz}\right)\left|_{z=L}&=-Ku_z\|_{z=L}\
\end{align*}
Then we get that 
\begin{gather*}
C_1=-\frac{F_0}{EA\kappa} \frac{Re^{-2\kappa L}}{1+e^{-2\kappa L}}\\
C_2= \frac{F_0}{EA\kappa} \frac{1}{1+e^{-2\kappa L}}
\end{gather*}
with 
\begin{align*}
R=\frac{K-EA\kappa}{K+EA\kappa}
\end{align*}
\newpage

\chapter{Linear Waves }
\section{Equations of motion}
Let's assume an hydrostatic vertical pressure distribution and constant density,
From the conservation of momentum equation in 1D we get\\
\begin{eqbox2}{}{Unitary discharge}
$$q(s,t)=\int_{z_b}^{h(t)} v_s\, dz$$
\end{eqbox2}
If we extend it to the whole cross-section 
\begin{eqbox2}{}{Cross section discharge}
\begin{align*}
Q(s,t)=&\int_{A(t)} v_s\, dA
\end{align*}
\end{eqbox2}

Using the conservation of mass (volume if we consider incompressible flow) we get 
\begin{align*}
\frac{\partial A(s,t)}{\partial t}+\frac{\partial Q(s,t)}{\partial s}&=0
\end{align*}

In channel flow, our main concern is the piezometric height ($h$), therefore is useful to use the following expression

\begin{align*}
\frac{\partial A(s,t)}{\partial t}&=\frac{\partial A}{\partial h}\frac{\partial h}{\partial t}=B(s,t)\frac{\partial h}{\partial t}
\end{align*}

If we consider small flow velocities, and/or small, semi-enclosed areas
\begin{eqbox2}{}{Small-basin approximation}
\begin{align*}
\frac{\partial h}{\partial s} \approx 0\longrightarrow Q(t)\approx A_b \frac{dh_b}{dt}
\end{align*}
\end{eqbox2}
 
As we saw in the continuum mechanics chapter, the integral version of the conservation principle is more general and powerful that the differential one, but in most cases, assuming a number of symmetries, the differential conservation is more useful. Nevertheless, we will briefly consider the integral case. Lets start with the continuity equation for permanent flow in 1D, and integrate it in the vertical direction
\begin{gather*}
\int_{z_b}^{h(t)}\left(\frac{\partial \rho}{\partial t}+\rho \nabla v\right)=0 \\
\int _{z_b}^{h(t)}\frac{\partial \rho}{\partial t}+\int_{z_b}^{h(t)}\frac{\partial \rho v_x}{\partial x}+\int_{z_b}^{h(t)}\frac{\partial \rho v_z}{\partial z}=0
\end{gather*}


Using the Leibniz integration rule

\begin{align*}
\left[\frac{\partial}{\partial t} \int\limits _{z_b}^{h(t)} \rho\, dz - {\color{cyan}\left(\rho \frac{\partial h}{\partial t}\right)_{z=h}}+{\color{orange}\left( \rho \frac{\partial z}{\partial t}\right)_{z=z_b}}\right]+
 \left[\frac{\partial}{\partial x}\int\limits_{z_b}^{h(t)} \rho v_x \,dz- {\color{cyan}\left(\rho v_x \frac{\partial h}{\partial x}\right)_{z=h}}+{\color{orange}\left( \rho v_x \frac{\partial z_b}{\partial x}\right)_{z=z_b}} \right]\\
 +\left[{\color{cyan} (\rho v_z)_{z=h}}+{\color{orange} (\rho v_z)_{z=z_b}}\right]=0
\end{align*}
The colored terms are the kinematic condition in the  boundaries, in particular, the {\color{cyan}cyan} on the surface and the {\color{orange} orange } terms  in the channel bed.
Re-ordering we get 
\begin{eqbox}
\begin{gather*}
\frac{\partial}{\partial t} \int\limits _{z_b}^{h(t)} \rho\, dz+\frac{\partial}{\partial x}\int\limits_{z_b}^{h(t)} \rho v_x \,dz= 
 {\color{cyan}\left[\rho\left( \frac{\partial h}{\partial t}+ v_x \frac{\partial h}{\partial x}+ v_z\right)\right]_{z=h}}
-
{\color{orange}\left[\rho\left(\frac{\partial z}{\partial t}+ v_x \frac{\partial z_b}{\partial x}+ v_z\right)\right]_{z=z_b} }\\
\frac{\partial}{\partial t} \int\limits _{z_b}^{h(t)} \rho\, dz+\frac{\partial}{\partial x}\int\limits_{z_b}^{h(t)} \rho v_x \,dz= 0
\end{gather*}
\end{eqbox}
The kinematics conditions terms are zero in both boundaries.
\begin{note}
The kinematic conditions are zero if we consider them to be \textit{streamlines}, intuitively we can see that 
the kinematic condition terms that involve temporal derivatives are zero because we assumed that the velocity normal to the flow is zero for the surface and seabed, the other terms relate to the fact that we assume that  a fluid particle approaching the seabed will move alongside it (cannot go into the seabed) and a fluid particle moving towards the surface will move along the surface ( not splash-out). \\
The density assumption is usually adequate for canals, but may not appropriated in the presence of temperature/salinity gradients.
\end{note}
Considering that the density is uniform throughout the cross-section 
\begin{eqbox2}{}{Mass Balance equation}
\begin{gather*}
\frac{\partial}{\partial t} \int\limits _{z_b}^{h(t)} \rho\, dz+\frac{\partial}{\partial x}\int\limits_{z_b}^{h(t)} \rho v_x \,dz= 0\\
\frac{\partial (h(t)-z_b)}{\partial t}+\frac{\partial q}{\partial x}=0\\
\frac{\partial d(t)}{\partial t}+\frac{\partial q}{\partial x}=0\\
\end{gather*}
\end{eqbox2}
Similarly to mechanics of solids , using  the mass balance and the conservation of momentum we get 
\begin{eqbox2}{}{Euler equation}
For incompressible, uniform density,negligible viscous stresses, and only gravity as an external force we get 

\begin{align*}
\frac{\partial v_i}{\partial t}+v_j\frac{\partial v_i}{\partial x_j}+\frac{1}{\rho}\frac{\partial p}{\partial x_i}=g_i
\end{align*}
\end{eqbox2}
So far, the equations of motion are formulated in a Cartesian coordinate system, but for geometries different than a straight canal, it can become algebraically tedious to work on ${\hat{x},\hat{y},\hat{z})$, therefore we introduce the natural coordinate system.
\begin{itemize}[]
\item $R$= Radius of curvature
\item $\hat{s} $ = Flow direction
\item $\hat{n} $ = Normal to the flow direction,towards the inside of the radius of curvature, defines the normal plane 
\item $\hat{b} $ = Bi-normal direction to the flow, defines osculation plane
\end{itemize}

\begin{eqbox2}{}{Euler equation in natural coordinates}
\begin{gather*}
\frac{\partial v_s}{\partial t}+v_s\frac{\partial v_s}{\partial s}+g\frac{\partial h}{\partial s}=0\\
\frac{\partial v_n}{\partial t}+\frac{v_s^2}{R}+g\frac{\partial h}{\partial n}=0\\
\frac{\partial v_b}{\partial t}+g\frac{\partial h}{\partial b}=0
\end{gather*}
\end{eqbox2}
\section{Long Waves}
Let's consider the case of a 1D \textit{long waves} with a constant cross section piezometric level, then we can assume that
\begin{itemize}
\item Hydrostatic pressure distribution $\longrightarrow \frac{\partial h}{\partial b}\approx0$
\item Negligible surface slope $\longrightarrow \frac{\partial h}{\partial n}\approx 0$
\item Acceleration only in $\hat{s}$
\item $\frac{\partial h}{\partial s}$ is constant within the cross-section
\item Considering cross-section averaged velocity $U=\int_A v \, dA/\int dA$
\end{itemize}
\subsection{Boundary Resistance}}
So far, we have only considered conservative forces, we can easily identify at least two types of non-conservative forces, exactly as we did with the master conservation principle, we have the volume force($\Sigma$) and the surface force($\phi$).
If we ignore for now the volumetric forces(turbulence,etc), and ignore the air-water interaction we get the frictional force of the canal walls/bed
(wet perimeter).
\begin{align*}
F_r=\tau_b P=\rho c_f|U|UP
\end{align*}
All the other forces have been expressed as force per unit mass, then 
\begin{align*}
\frac{F_r}{\rho A_c}=\frac{\rho c_f|U|UP}{\rho A_c}=c_f\frac{|U|U}{R}
\end{align*}
\begin{note}
This assumes that $c_f$ is constant in the wet perimeter, which is not always the case. See Orton's  and Lotter's method for compound cross-sections.\\
Also, see Manning and Chézy equation for other ways to estimate boundary resistance. Detail in \cite{ICH2304}
\end{note}
Then, the equation on the whole cross-section
\begin{eqbox2}{}{Saint-Venant Equation}
\begin{gather*}
B\frac{\partial h}{\partial t}+\frac{\partial Q}{\partial s}=0\\
\frac{\partial Q}{\partial t}+\frac{\partial}{\partial s}\left( \frac{Q^2}{A_c}\right) +gA_c\frac{\partial h}{\partial s}+c_f\frac{|Q|Q}{A_cR}=0
\end{gather*}
\end{eqbox2}
\section{Dimensional analysis}
We can characterize a wave by four parameter
\begin{itemize}
\item Wave length ($\lambda$ or $L_0$)
\item Wave Period ($T_0$)
\item Flow velocity ($U_0$)
\item Water depth ($h$ or $D_0$)
\end{itemize}
If we look at the Saint-Venant equation, by inspection is evident that the storage ($B\frac{\partial h}{\partial t}$) and volume transport($\frac{\partial Q}{\partial s})$ terms cannot be neglected. 


Lets consider the \textit{importance} of the local inertia and use it to normalize the effect of other terms in the momentum equation
\begin{eqbox}
\begin{gather*}
\text{Normalized local inertia}\propto\left(\frac{U}{T_0}\right)/\left(\frac{ U}{T_0}\right)= 1\\
\text{Normalized Advective Inertia}\propto \left(U\frac{U}{L}\right)/\left(\frac{U}{T_0}\right)=\frac{UT_0}{L}\approx Fr\\
\text{Normalized Gravity forcing}\propto \left(g\frac{H_0}{L_0}\right)/\left(\frac{U}{T_0}\right)= \frac{gH_0T_0}{L_0U_0}\\
\text{Normalized Resistance}\propto \left(c_f\frac{U_0^2}{D_0}\right)/\left(\frac{ U}{T_0}\right)=c_f\frac{U_0T_0}{D_0}
\end{gather*}
\end{eqbox}
In general the adjective inertia and/or resistance can be neglected, is then useful to define the ratio between them
\begin{eqbox}
\begin{align*}
\sigma = c_f\frac{U_0}{D_0}\frac{T_0}{2\pi}=c_f\frac{U_0}{\omega D_0}
\end{align*}
\end{eqbox}
Let's define the velocity of a wave relative to the flow as
\begin{eqbox2}{}{Celerity}
For shallow water the propagation speed (phase velocity) can be estimated as 
\begin{align*}
c_p\approx\sqrt{gh}
\end{align*}
For deep water, $\frac{D_0}{L_0}>0.5}$
\begin{align*}
c_p\approx \frac{g}{2\pi}T
\end{align*}
\end{eqbox2}
\section{Classification/Characterization of long waves}
In practice we consider a wave as \textit{long} if $\frac{L_0}{D_0}> 20$.
{\color{magenta}\subsection{Translatory waves}}
These waves are formed when control structures, such as gates, valves, etc are manipulated.

{\color{magenta}\subsection{Tsunami wave}}
Impulsive wave to due tsunami. Can be produced to either  an mega-rupture tsunami or due to an earthquake-produced landslide. 
\begin{itemize}
\item $\lambda \sim  10^{2-3}$ km
\item $h\sim 10^{3}$ km
\item $T\sim 10-20 $ mins
\item Then, $c\sim 10^2 m/ s $
\item Then, $\sigma\sim 10^{-4}$
\end{itemize}

{\color{magenta}\subsection{Seiches}}
Natural oscilation in lakes and basins, due to geometric, gravitational and atmospheric variations. 
\begin{itemize}
\item $\lambda \sim  20$ km
\item $h\sim 10$ m
\item $T\sim 20 $ mins
\item Then, $c\sim 10 m/s$
\item Then, $\sigma\sim 10^{-2}$
\end{itemize}

{\color{magenta}\subsection{Tides}}
Produce to gravitational effects of the moon and sun.
In the ocean
\begin{itemize}
\item $\lambda \sim  8-9 \times 10^3$ km
\item $h\sim 3-5 \times 10^3$ km
\item $T\sim 745 $ mins
\item Then, $c\sim 200 m/s$
\item Then, $\sigma\sim 10^{-3}$
\end{itemize}
The resistance factor of tides in coastal areas, channels, tide flats could be in the order of $\sigma\sim [10^{-1},10^{1}]$ 

{\color{magenta}\subsection{Flood waves}}
Cause by excessive precipitation or sudden melting of snow.
\begin{itemize}
\item $\lambda \sim   10^{2-3}$ km
\item $h\sim 1-2\times 10$ m
\item $T\sim 3-5\times 10^3 $ mins
\item Then, $c\sim 7 m/s$
\item The effect of bed friction is usually considerable in the order of $ \sigma \sim 5 \times 10^{1}$
\end{itemize}

{\color{magenta}\section{Wave propagation}}
Let's consider a canal with a gate that separates two piezometric levels. If we open the gate, a translatory wave will propagate, the water will flow from the \textit{higher} to the \textit{lower} level , creating a positive amplitude wave in the \textit{lower} side and a negative amplitude in the \textit{higher} side.

We will define the amplitude of the wave (respect to the piezometric level at $t_0$) as $\zeta$\\
\begin{eqbox2}{}{Simplified wave propagation}
Lets consider a control volume such that 
$$h(s_1,t)=h_0+\zeta \quad || \quad  h(s_2,t)=h_0$$
Using the  mass conservation principle (volume, if we consider $\rho$ to be constant)
\begin{align*}
B\frac{\partial h}{\partial t}+\frac{\partial Q}{\partial s}\approx&  B\frac{\zeta}{\Delta t}+\frac{Q}{\Delta s}\\
\approx & B\zeta\Delta s = Q\Delta t \longrightarrow   Q=Bc\zeta
\end{align*}
Then, for the momentum equation ignoring the resistance and advective acceleration terms
\begin{align*}
\frac{\partial Q}{\partial t}+gA_c\frac{\partial h}{\partial s}\approx&\frac{ Q}{\Delta t}+gA_c\frac{\Delta h}{\Delta s}\\
\approx& \frac{ Q}{\Delta t}+gA_c\frac{\zeta}{\Delta s}
\longrightarrow gA_c\zeta=cQ
\end{align*}

\begin{note}
Considering a rectangular cross section we obtained the expression for the celerity that we mentioned before
\end{note}

\end{eqbox2}
Lets re-examine the Saint-Venant equation
\begin{align*}
B\frac{\partial h}{\partial t}+\frac{\partial Q}{\partial s}=0
\end{align}
\begin{align*}
\frac{\partial Q}{\partial t}+\frac{\partial}{\partial s}\left( \frac{Q^2}{A_c}\right) +gA_c\frac{\partial h}{\partial s}+c_f\frac{|Q|Q}{A_cR}=0
\end{align*}
For low and long waves we can neglected advective acceleration and boundary resistance term
\begin{align*}
\frac{\partial}{\partial s}\left( \frac{Q^2}{A_c}\right) \approx 0\\
c_f\frac{|Q|Q}{A_cR}\approx 0
\end{align*}
Assuming a constant rectangular cross-section and combining the equation we get the \\
\begin{eqbox2}{}{Elementary wave equation}
\begin{align*}
\frac{\partial^2 h}{\partial t^2}-\frac{gA_c}{B}\frac{\partial^2h}{\partial s^2}=0\\
\frac{\partial^2 h}{\partial t^2}-c^2\frac{\partial^2h}{\partial s^2}=0\\
\end{align*}
\end{eqbox2}
\section{General solution: d'Alembert}
The general solution to the elementary wave equation is 
\begin{eqbox2}{}{d'Alembert's solution}
\begin{align*}
h(s,t)=h^+(s-ct)+h^-(s+ct)\\
h(s,t)=h^+(S^)+h^-(S^-)\\
\end{align*} 
Sometimes, for situation with non-homogeneous boundary conditions, its better to use the following formulation
\begin{gather*}
h(s,t)=h^+(t-s/c)+h^-(t+s/c)\\
\end{gather*} 
\end{eqbox2}
The solution is the superposition of two waves traveling at a constant speed in opposite directions.
Because the wave has no distortion, it  \textit{simply} propagates with fixed velocity, then the value of $h$ is constant in $S$ , or alternative $s=ct+constant$\footnote{See, method of characteristic for more detail \hl{put here hyper-reference to this section}}

Then it may be useful to define to total derivative as 
\begin{gather*}
\frac{dh}{dt} =\frac{\partial h}{\partial t}+V\frac{\partial h}{\partial s}
\end{gather*}
If consider the case of our propagating wave 
we get that the total derivative of $h$ is zero
\begin{gather*}
\frac{dh^+}{dt} =\frac{\partial h^+}{\partial t}+c\frac{\partial h^+}{\partial s}\\
\end{gather*}
This relationship  is very useful because it lets us relate  the temporal and spatial derivative of $h$ 
\begin{gather*}
\frac{\partial h^+}{\partial x}=-\frac{1}{c}\frac{\partial h^+}{\partial t}
\end{gather*}

\subsection{Boundary conditions}
We will consider common boundary conditions that happen canals,however they can be easily related to boundary condition of propagating wave in a bar as it was shown in Assignment 2 ($h\sim \varepsilon_x$ and $Q\sim N$)\\

\textbf{{\color{magenta}Canal with close end}}\\

As the wave approaches the end of the canal, locally 
\begin{align*}
Q=\delta Q^++\delta Q^- =Bc(\delta h^+-\delta h^-)=0
\longrightarrow \delta h^+= \delta h^- 
\end{align*}
We know that the solution is the superposition of the waves traveling in opposite directions then  we get \textit{fully positive reflection}
Which is equivalent to the case of a  semi-infinite bar restricted at the end


\begin{align*}
h=\delta h^++\delta h^-=2\delta h^+
\end{align*}
\textbf{{\color{magenta}Canal connected to reservoir}}

As the wave approaches the reservoir , locally 
\begin{align*}
\delta h=\delta h^++ \delta h^-=0\longrightarrow \delta h^+=- \delta h^-
\end{align*}
We know that the solution is the superposition of the waves traveling in opposite directions then  we get \textit{fully negative reflection}.\\
Which is equivalent to the case of a  semi-infinite bar restricted at the end\\
\textbf{{\color{magenta}Canal with rapidly varying cross-section}}\\
The transition is characterized by a discontinuous cross-section.

Qualitatively we can expected \textit{partial reflection}, and that is exactly what we get.

Defining compatibility conditions in the transition
\begin{align*}
\text{Continuous water level at the transition}&\longrightarrow\delta h_t=\delta h_i+\delta h_r\\
\text{Continuous discharge at the transition}&\longrightarrow\delta Q_t=\delta Q_i+\delta Q_r
\end{align*}
We define the following adimensional parameters
\begin{gather*}
r_t=\frac{\delta h_t}{\delta h_i}\\
r_r=\frac{\delta h_r}{\delta h_i}\\
\gamma=\frac{B_2c_2}{B_1c_1}=\sqrt{\frac{A_{c,2}B_2}{A_{c,1}B_1}}
\end{gather*}
Then, the compatibility conditions are expressed as 
\begin{gather*}
r_t=1+r_r\\
\gamma r_t=1-r_r
\end{gather*}
Then, the reflected and transmitted wave ratios are 
\begin{eqbox}
\begin{align*}
r_r=&\frac{1-\gamma}{1+\gamma}\\
r_t=&\frac{2}{1+\gamma}=1+r_r
\end{align*}
\end{eqbox}
\textbf{\color{magenta}{Multiple Channels}}\\
If the abrupt transition consist of multiple channel, we can extend the previous principle, we define an \textit{equivalent channel} such that 
\begin{eqbox}
\begin{gather*}
B^*c^*=\sum B_ic_i
\end{gather*}
\end{eqbox}
Then applying conservation of mass and momentum in the transmitted channel, we determine the height/discharge in each channel
\begin{note}
We are assuming that the wave is \textit{insensitive to changes in propagation direction}. 
Also, see example of how how having a side close channel the affect the reflection int the transmitted and original channel, in  \cite{unsteady} p.50.
\end{note}
\textbf{{\color{magenta}Gradually varying cross-section}}\\
This boundary conditions escapes the scope of the course, see chapter 4.3.2 of \cite{unsteady} for  basic explanation.
\end{itemize}
\section{Periodic Waves}
In the previous section we considered the case of a single wave, let it be a translatory wave or a single wave, lets extend the same concepts to a periodic wave.\\
\begin{eqbox2}{}{Periodic wave}
\begin{align*}
\zeta^+(s,t)=&\hat{\zeta}\cos \left(\frac{2\pi}{L} (s-ct)\right) \\
\zeta^+(s,t)=&\hat{\zeta}\cos(ks-\omega t)
\end{align*}
With: 
\begin{align*}
\text{Wave number}(k) &:=\frac{2\pi}{L}\\
\text{Angular frequency}(\omega) &:=\frac{2\pi}{T}
\end{align*}
\end{eqbox2}
\subsection{Boundary conditions}
\textbf{{\color{magenta}Canal with closed end}}\\
The boundary condition is 
$$Q=\delta Q^++\delta Q^- $$
Using the d'Alembert method, considering the boundary conditions
\begin{gather*}
\zeta=\zeta^++\zeta^-\\
\zeta=2\hat{\zeta}\cos(ks)\cos(\omega t)
\end{gather*}\\
\textbf{{\color{magenta}Closed basin}}\\
The boundary condition is 
$$ Q=0$$
Then the antinodes of the wave are on the boundaries, then 
\begin{gather*}
k_n=\frac{n\pi}{\ell}
\end{gather*}

Because we are dealing with un-forced oscillations, the natural frequencies are 

\begin{gather*}
\frac{\omega_n}{k_n}=\sqrt{\frac{gA_c}{B}}\\
\omega_n=k_n\sqrt{\frac{gA_c}{B}}
\end{gather*}\\
\textbf{{\color{magenta}Semi-closed basin connected to reservoir}}\\
At the close end 
$$ Q=0$$
and at the reservoir
$$ \zeta=0$$
Then, to fulfill this condition, at all times there is a antinode in the closed end, and a node in the reservoir end.
\begin{gather*}
k_n\mathcal{l}=\frac{1}{2}\pi+n\pi
\end{gather*}
And the natural frequencies are  

\begin{gather*}
\omega_n= \left(\frac{1}{2}\pi+n\pi \right)\frac{1}{\ell}\sqrt{\frac{gA_c}{B}}
\end{gather*}\\
\textbf{{\color{magenta}Semi-closed basin connected to tidal see}}\\
At the close end 
$$ Q=0$$
and at the sea-basin interface
$$ \zeta(\ell,t)=\zeta_{sea}$$
The oscillation in the basin are forced by the oscillation of the tide, then
\begin{gather*}
\zeta(0,t)=\frac{\zeta(\ell,t)}{|\cos k\ell|} $$
\end{gather*}
Then we get resonance when
\begin{gather*}
\cos k\ell\rightarrow 0 \Longrightarrow \zeta(0,t)\rightarrow \infty
\end{gather*}
For small basins,
$$(\ell\ll L \quad \text{or} \quad  k\ell\ll 2\pi)\Longrightarrow \cos(k\ell)\approx 1$$
Then \textit{"the water level responds almost in unison to the tidal forcing at
the mouth, rising and falling with the tide but being virtually horizontal at all times"}(\cite{unsteady}), this phenomenon is called {\color{magenta}Pumping or Helmholtz mode}
\section{Case Study: Dispersive wave in foundation pile}
We previously derived the equation of motion for the bar 
\begin{eqbox2}{}{ "Telegraph  equation"}
\begin{gather*}
\frac{1}{c^2_0}\frac{\partial^2u_z}{\partial t^2}-\frac{\partial u_z}{\partial z^2}+\kappa_d^2u_z-q_z=0\\
c^2_0=\frac{E}{\rho}\\
\kappa^2_d=\frac{\xi}{EA}=\frac{\xi}{c^2_0\rho A}
\end{gather*}
\end{eqbox2}
This wave equation is dispersive, therefore we cannot use the d'Alambert method to solve it. 
Lets consider a SDOF system
\begin{eqbox2}{}{SDOF: Undamped, Harmonic Excitation}
We know that the response of a undamped system, due to an harmonic excitation is simply the modulated signal shifted in time, or simply

\begin{gather*}
u(t)=Re[\tilde{u}(\omega)\exp(i\omega t)]
\end{gather*}
with 
\begin{gather*}
\tilde{u}=\hat{u} e^{i\alpha}\\
\hat{u}=\vert \frac{1}{k-m\omega^2}\vert \hat{F}_0\\
\end{gather*} 
\[\alpha = \begin{cases} 
          0 & \omega\leq \sqrt{\frac{k}{m}} \\
          \pi &\sqrt{\frac{k}{m}}\leq \omega \\
       \end{cases}
    \] 
\end{eqbox2}
Replacing in the original equation we get 
\begin{gather*}
\frac{d^2\tilde{u}}{dz^2}+\left(\frac{\omega^2}{c_0^2}-\kappa_d\right)\tilde{u}=0
\end{gather*}
The solution of the ODE is known to be 
\begin{gather*}
\tide{u}(z,\omega)=C_1e^{-i\gamma z} +C_2e^{(i\gamma z)}\\
\gamma=\frac{1}{c}\sqrt{\omega^2-(c\kappa)^2}=\frac{1}{c}\sqrt{\omega^2-\omega_c^2}
\end{gather*}
The response depends on the value of $\gamma$, let's suppose that 
\[\gamma = \begin{cases} 
          k  & \omega_c\leq \omega \\
          -i\mu &\omega\leq \omega_c \\
       \end{cases}
    \] 
    
Then we can characterize the spatial response in term of $\gamma$, if 
\begin{itemize}
\item $\gamma=k$ \\
The solution is a propagating harmonic wave
$$u_z(s,t)\sim \tilde{u}_z(s,t)e^{i\omega t}=C_1e^{i(\omegat t-kz)}+C_2e^{i(\omega t+kz)}$$
\item $\gamma=-i\mu$ \\
The solutions is a evanescent wave 
$$u_z(s,t)\sim \tilde{u}_z(s,t)e^{i\omega t}=C_1e^{-\mu z}e^{ i\omega t}+C_2e^{-\mu z}e^{i\omega t}$$
\end{itemize}
This is analogous to the damped/critically damped/over-damped behavior that we are familiar in Structural Dynamics. 
Lets consider to different boundary conditions
\subsection{Boundary Conditions: Infinite Pile}
We can easily deduce two boundary conditions. First, the Normal force is the "top face" is opposite to the driving force by simple summation of force. Second, because the pie is infinite the deformation at the end is null ( radiation condition), mathematically 
\begin{gather*}
EA\frac{\partial u}{\partial z}\lvert_{z=0}=-Re\{\hat{F}_0 e^{i\omega t}\} \\
u_z\lvert_{z=\infty}=0

\end{gather*}

\newpage
\newpage

\chapter{Diffusion: Heat Conduction}
Using the master conservation equation for the case of energy, if we only consider internal energy (in this case temperature)
\begin{eqbox}
\begin{gather*}
\frac{\partial Q}{\partial t} =-\nabla \cdot q
\end{gather*}
Considering that the "Volumetric Storage Capacity"($Q$) is 
\begin{gather*}
Q=c\rho T
\end{gather*}

$c$: Specific heat capacity\\
$\rho$: Density\\

\end{eqbox}
And considering that the flow of heat is proportional to the temperature gradient (conduction)
\begin{eqbox2}{}{Fourier's Law}
\begin{gather*}
q=\alpha\nabla T
\end{gather*}
$\lambda$: Thermal conductivity
\end{eqbox2}
If we ignore advection, radiation and convection mechanisms, we can define the heat diffusion equation

\begin{eqbox2}{red}{Simplified Heat Transfer or Heat Diffusion}
\begin{gather*}
\frac{\partial T}{\partial t} =\alpha\nabla^2T\\
\end{gather*}
\end{eqbox2}
Solving the equation is more difficult than the continuity equation, but we can still find an analytic solution
\section{Homogeneous BC}
Lets consider homogeneous boundary conditions,and know initial value, and lets assume that 
\begin{gather*}
T(x,t)=v(x)w(t)
\end{gather*}
Then 
\begin{gather*}
v(0)w(t)=0 \text{ only if } v(0)=0\\
v(L)w(L)=0 \text{ only if } w(L)=0\\
\end{gather*}

If we replace the new expression for $T$ in the differential equation and separate the spacial from the temporal variables
\begin{gather*}
\frac{\partial T}{\partial t} =\alpha\nabla^2T\\
\frac{1}{\alpha w(t)}\frac{\partial w(t)}{\partial t}=\frac{\nabla^2 v(x)}{v(x)}
\end{gather*}
Because the equation is linear and we separate variable we can \textit{say} that the temporal variables equals to a constant, and the spacial variable to the same constant(P), where P does not depend neither on $t$ or $x$
\begin{gather*}
\frac{1}{\alpha w(t)}\frac{\partial w(t)}{\partial t}=-P\\
\frac{\nabla^2 v(x)}{v(x)}=-P
\end{gather*}

Using separation of variable, for the temporal part we get that 
\begin{gather*}
w(t)=e^{-\alpha P t}e^C_1
\end{gather*} 
and for the spatial part 
\begin{gather*}
v(x)=C_2 \cos(\sqrt{Px})+C_3\sin(\sqrt{Px})
\end{gather*}

\newpage

\bibliography{references.bib}

\newpage
\end{document}

\newpage




















We may be interested in the rate at which the excess water pressure is dissipated and also the amount of consolidation.\\

\hl{put here derivation of 1D consolidation eq}
\begin{eqbox2}{red}{Terzaghi's 1D Consolidation}
\begin{align*}
\frac{\partial\text{u}_w}{\partial t}
=C_v\left(\frac{\partial^2\text{u}}{\partial z^2}\right)
\end{align*}
\end{eqbox2}
\begin{note}
To quantify the amount of primary and secondary consolidation with numerical and graphical method  or the effect of, (over)consolidated clays, ramp loading correction, $C_v$ estimations,etc. See ICE 2604 and ICE 2614)
\end{note}




